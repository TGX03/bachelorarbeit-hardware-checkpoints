\chapter{Foundations}\label{chap:Foundations}
The purpose of this chapter is to lay the groundwork for this thesis
and to establish the technical terms.
In addition, contradicting definitions are explored
and the process of deciding which ones will be used is explained.
The basic technologies involved in this thesis are also addressed in this chapter.

\section{Emulation \& Simulation}
When discussing the internals of an emulator, one first needs a clear definition of this term,
and to distinguish it from the term of \emph{simulation}, which may appear next to it.
When using the naive approach and simply searching for such a definition using Google,
the second and third results are 2 questions on StackOverflow\cite{SO_link}\cite{SO}.
\footnote{See \autoref{fig:Google_SO}.}
Both these questions have at least 10 answers and multiple 100 upvotes,
which means they carry some relevance in the non-scientific IT community.
Under the answers of one of the posts, another commentator replies
\enquote{Exactly the opposite answer here:}\cite{SO_link},
linking to the other post\cite{SO},
as indeed the top answers of these posts, as well as the replies on each post,
are in direct conflict with each other.
As these discrepancies were mentioned,
a small discussion commences and the involved parties seem to reach the conclusion
of the definition being not entirely clear in their community and it should likely be precisely defined when talking about it
to not create confusion.

The purpose of this small example is to yield a conclusive definition into this thesis,
but to illustrate the necessity of clearly defining these terms for this thesis.
To achieve this, information about existing definitions was researched in 2 different approaches,
the first one outside the scientific community by looking at more generalized definitions from the Oxford dictionary
as well as by looking at the usage of these terms in the world of video games,
the second one directly searches for a definition inside the scientific community and its research papers.

\subsection{Emulation \& Simulation outside the scientific community}
The Oxford English Dictionary defines the 2 terms as follows:
\begin{itemize}
    \item \emph{Emulation}: \enquote{The technique by which a
    computer is enabled, by means of special
    hardware or software, to execute programs
    written for a different type of computer}\cite[p. 196]{emulation_oxford}.
    \item \emph{Simulation}: \enquote{The technique of imitating the behaviour of
    some situation or process (whether economic,
    military, mechanical, etc.) by means of a
    suitably analogous situation or apparatus, esp.
    for the purpose of study or personnel training}\cite[p. 503]{simulation_oxford}.
\end{itemize}
From these definitions, one may deduce that emulation is a special case of simulation,
in which an \enquote{apparatus for the purpose of study} is created to \enquote{imitate the behavior of some situation or process},
namely the internal workings of a computer of a different type.
The term of \enquote{simulation} instead is easily scaled up or down to suit the current needs.
It may describe a device imitating a large software infrastructure running across many different devices,
perhaps by emulating computers involved in such a system,
however it may also describe the process of imitating the memory management unit of a CPU inside an emulation.
This may be concluded by saying that emulation is a special type of simulation,
which itself requires smaller systems of simulation to function,
but can then itself be used to create bigger systems of simulation.

\subsubsection*{Emulation \& Simulation in video games}
A large field in which emulators and simulators are used and accessible to ordinary people
is the world of video games,
where they however fulfill very different purposes.
Simulators are usually a type of game,
which allows the player to have an experience they usually do not have in real life.
Simulators in this field try to make the user feel as if the imitation they are experiencing is very realistic,
even though it is usually simplified heavily to still allow for high accessibility.
An example are flight simulators, in which many players feel like they are getting a realistic experience,
even though many features like navigation or communication with Air Traffic Control are heavily simplified
and can even be turned off.
Which is also an important feature of simulators, the ability to tune the precision
to the current needs to produce a result that is both precise enough for the current problem,
while not wasting resources during unnecessary calculations.\linebreak
Emulators on the other hand get used to play games which were developed for systems that no longer exist.
An example of such an emulator is xemu\cite{xemu},
which is also used later on in \autoref{chap:QEMU_API} to demonstrate the interaction with QEMU.
Such emulators \emph{simulate} the physical hardware of a specific game console
as closely as necessary for the user to play a specific video game at a reasonable performance.
Xemu for example has to emulate the original Intel Coppermine CPU and Nvidia MCPX X3 GPU
for the games to function, and it must emulate features that may not be needed at all.
For example, it always must emulate the SIMD registers of the CPU, even though a game may not need them.
An additional constraint is performance, emulators usually must run at the same speed as the original device,
and any potential performance benefit a modern computer offers is therefore void.
This results from the games being developed for a system which offers a specific performance,
and when run on a faster system, the game, especially its physics simulations,
runs faster in accordance with the performance gain of the emulating system,
which would usually make the game unusable.

\subsection{Emulation \& Simulation in scientific literature}
In scientific literature, these terms are often not clearly defined, but instead the reader is often assumed to be somewhat familiar with both terms.
An example of a strict definition of both terms is the one formulated by \Citeauthor{definition_iot}
in a paper about the future chances of the Internet of Things for logistics\cite{definition_iot}.
They explicitly define emulation as a special case of simulation,
and the definitions are generally compatible with the one before.
\begin{itemize}
    \item \emph{Simulation}: A simulator only models and abstracts the system it is simulating.
    This means, in its development decisions must be made as to how precise the simulator works.
    For this, a model must be created and requirements for actual execution developed.
    Later on, the results of such a simulation must then be carefully evaluated to make sure
    there are no errors that come from such abstraction or are accounted for and compensated.
    The advantage of such a process is the high speed it carries,
    as simplifications often take away lots of required computation time.
    The disadvantage, as mentioned, however is the potentially high error rate for results,
    and the simulation usually breaks down when trying to analyze small details of a larger system\cite{definition_iot}.
    \item \emph{Emulation}: An emulator is heavily coupled to real components.
    In addition, its field of work is usually smaller and defined very precisely.
    In return, the results of an emulator are very precise, in most cases exact.
    Error states of the real hardware are also modeled, meaning when feeding it with incorrect data,
    one can observe how the system will behave in abnormal states, which often isn't modeled in a simulation.
    The big disadvantage of emulation however is its massive cost,
    both in development time and resources,
    as emulation usually requires a lot more computing power than the device it emulates\cite{definition_iot}.
\end{itemize}

\subsection{Conclusion}\todonote{Think of a better title}\label{sec:conclusion_emulation}
In this thesis therefore emulation will be viewed as a special kind of simulation,
which simulates a computing device of a specific architecture to a very high level of detail
on another computing device, which may or may not be of another architecture.
This especially distinguishes emulation from software like Wine, which is not an emulator\todonote{I really need to stop},
as such software only provides other runtime software a program requires to function, but does not simulate another device in any way\cite{wine}.

In contrast, the scalability of Simulation is much wider,
especially as a simulator may leave out aspects of the real system
its developers don't deem as neccessary for the experiment they want to conduct.
An example of such software in software development is Mockito,
which provides to other parts of a program certain APIs during testing,
which however are not fully implemented yet,
but still return plausible values that allow for testing.

\section{Full System Simulation}
A term which also often appears around emulation is "full system simulation".
The definition for this is mostly taken from \Citeauthor{kitcheckpoints}\cite{kitcheckpoints}.\todonote{Maybe rephrase this to not make it appear like plagiarism.}

When developing a system, which also requires the development of an operating-system-like
software, the actual hardware may not yet be available for testing
as it often is still in development itself.
Especially when developing embedded devices,
the code developed for it often runs on a very low level
and the kernel may take on additional tasks it doesn't on general-purpose computers.
To allow for testing of any developed software,
the devices it's supposed to run on must be simulated in its entirety,
with its CPU, RAM, GPU, and other external components behaving exactly as they would on the actual device.

Especially such special external components are an issue,
as most emulators provide simplified APIs that are adapted to them being used by a general-purpose computer.
For example, Hyper-V on Windows does not have the ability to provide a GPU to the guest,
instead Linux has to fallback to \emph{llvmpipe} which provides software rendering.
Other software like VMware and VirtualBox provide heavily simplified GPUs to a guest system,
which are often unable to process 3D rendering by default,
and such a feature must be manually enabled if required.
But even then, there is no option to emulate an actual GPU which may be used on a physical device.

This definition seems very similar to actual emulation,
however, full system emulation is a broader term, under which emulation is a specific technique to achieve it.
Other full system simulations may approximate the internals of a CPU or other parts of a computer,
which is something not allowed for emulation.

\subsection{Anomalies in hardware}
An important part of such full system simulation is the testing of the behavior of code on the actual hardware.
From time to time code written in languages like C exerts behaviors contradicting the official specification
once it is executed on actual hardware.
A famous example of such an anomaly is bug 323\cite{323} in gcc,
which, while first reported in 2000, remains unresolved to this day.
This bug causes the result of a floating point comparison to change depending on when it is executed
and whether the compiler decides to write a result from the registers into system memory.

Such bugs can only be caught by testing on the real system
or by having a very detailed emulator running the code.
In the case of bug 323, only an emulator that accurately simulates 80-bit floating point registers
as well as the conversion to 64-bit values once the value gets written to memory
would be able to reproduce this bug.
Any emulator using only 64-bit floats for internal calculations would not be able to reproduce this bug.

\section{Emulation Software}\label{sec:emulators}
As mentioned in \autoref{sec:conclusion_emulation}, emulators are often used in the world of video games.
These highly specialized emulators get used to correctly simulate the internals of a given video game console
which allows the usage of video games developed for this game console on a different device or general-purpose computer.
However, such highly specialized emulators are rarely useful for actual research,
as any kind of simulated hardware cannot be easily swapped for something else without modifying the code of the software.

For such tasks, general-purpose emulators are also available.
Such software can be separated into 2 kinds of software:

\subsection{Virtualization \& Virtual machines}
When not talking about video games, Hyper-V and especially VMware are names often associated with this topic,
as they represent a large share of the software used in data centers\cite{VMware}.
However, such tools usually do not fall under the definition of \emph{emulation},
but instead they perform a task called \emph{virtualization}.
The goal of such software is not to emulate a different architecture from the one the program is run on,
but instead to create an environment isolated from the host and other operating systems in which an operating system can still
be executed as if it were running on actual hardware.
To allow for an efficient execution of this task, modern CPUs also include specific hardware for virtualization,
on Intel it is called VT-x and AMD calls it AMD-V.
These technologies allow for the execution of such isolated code with near-native speed,
meaning there is nearly no performance penalty while effectively having 2 systems running on the same machine.
For example these technologies are employed in large data centers,
where on a single physical server many different systems, usually from different clients,
are hosted and completely isolated from one another.
This reduces cost and overhead, as not every client requires its own physical server,
especially if clients only have very small performance requirements and require only a single CPU core,
and perhaps even then use only a small part of the available computing power.

An important part of virtualization is the so-called \emph{hypervisor},
which is responsible for isolating the guest systems from each other and the host system,
while providing necessary resources to the guests\cite{hypervisor}.
There are 2 different kinds of hypervisors which can be distinguished:

\paragraph{Type 1 hypervisor} A type 1 hypervisor runs directly on the physical system
and all other operating systems run on top of it.
This includes the operating system the user experiences as the host OS,
which only gets more rights and resources allocated to it from the hypervisor than other guests.
The advantage of such a hypervisor is less overhead and higher speed,
however it offers less flexibility as the hypervisor is directly tied to the operating system used.
KVM is the standard type 1 hypervisor included in Linux,
while the type 1 hypervisor on Windows is called Hyper-V\cite{hypervisor}.

\paragraph{Type 2 hypervisor} A type 2 hypervisor is executed inside a host operating system,
only getting allocated resources from said operating system.
This even allows for virtual machines to be run purely in user mode,
as well as portability between different operating systems,
under the assumption that the software used is available for multiple operating systems.
This makes type 2 hypervisors a lot more flexible,
while not having direct access to system resources,
which may carry a performance penalty.
Examples of such software include VirtualBox and VMware Workstation\cite{hypervisor}.
In addition, all emulators emulating a different architecture always are a type 2 hypervisor.

In addition, some virtualization also emulates other parts of the system,
for example graphics cards for 3D accelerations, sound cards, network cards
and other PCI devices.
As mentioned before, such emulation is usually not very detailed or capable,
and instead only provides the basic functionality associated with such a device.
VirtualBox for example has the option of emulating a device called \enquote{VMSVGA}
for the guest system, which implements basic rendering as well as the APIs
an operating system used to interact with it.
However it in no way resembles any actual GPU from Nvidia, AMD or Intel,
and is very limited in functionality.
As it is emulated by the host CPU, it is also a lot slower than a real dedicated GPU,
and its video memory is limited to 256MiB, while real GPUs are usually in the range of 10GiB.
Still, such features make software that offers them a hybrid between virtualization and emulation.

\subsection{Explicit emulators (QEMU)}
There are many different solutions available that offer emulation of specific architectures,
however just like emulators for game consoles, they are very restricted and not useful in a flexible environment.
To mention a single example, DOSBox emulates an Intel 286/386 processor on any hardware it is compiled for.
This may at first raise the question of why one would emulate an x86 CPU on an x86 CPU,
as this tool is mainly used on Windows,
however a modern x86 CPU running modern Windows performs vastly differently than
an x86 CPU running in real mode and at a speed of 8Mhz.

However, when it comes to "general purpose emulators",
meaning an emulator which can add or remove resources to suit the needs of the situation
and is even capable of emulating different architectures,
there currently exists only a single tool I am aware of that features these capabilities,
which is QEMU (Quick Emulator).
For this reason, QEMU will be used as the emulator software in this thesis,
as there are no other tools with the flexibility of QEMU available.
This point is especially proven as many other tools which offer emulation
use QEMU under the hood, for example Android Studio runs virtual devices using QEMU\footnote{See \href{https://developer.android.com/studio/run/emulator-commandline?hl=de}{https://developer.android.com/studio/run/emulator-commandline}},
and software like Lapidary\cite{lapidary} or Gem5\cite{gem5} also rely on it for their simulations.
QEMU also offers many resources on its internal workings and is fully open-source
which allows for easy inspection of its workings.
The research around it also offers many insights into its capabilities.

Most internals of QEMU will be explicitly discussed in \autoref{chap:QEMU_API},
as these technical details are both very important for this thesis
but very inconsistent in their technical documentation.

\section{Checkpoints}
When simulating a system, it is of course always possible to interrupt the simulation and freeze the current state of the simulator in memory.
From such a frozen and consistent state, it is now possible to extract certain data
for other uses separate from the simulation itself.
When permanently stored, such data is generally referred to as a \enquote{checkpoint},
and to research the ways of creating such a checkpoint from a running QEMU instance is the aim of this thesis.
It is desired to process the data gathered in a way that makes it accessible to both machines and humans,
as a simple memory dump from the emulation software is not helpful to the human analyzing it,
and it contains overhead data from the emulator which would ideally be stripped
to only analyze the data gathered with regard to the actual hardware executing it in the real world.

In this thesis, checkpoints reference a copy of the complete internal state of a hardware architecture at a specified point in time.
For the purposes of this thesis, this references all CPU registers, flags, system memory,
storage devices and recursively the state of external devices of a computer.
This thesis will later on also discuss possible approaches for reinjecting such a checkpoint
into a running emulation, however implementing such a process is not the goal of this thesis.

When modeling a full software infrastructure,
such checkpoints can be used to supplement a model of individual states of the system as a whole,
from which detailed behavior of the system may be deduced
as well as other issues like performance addressed.
It may also be helpful to communicate with developers of other parts in the system.
Especially for error states, such checkpoints are very useful,
as it may not always be possible to create an error state manually,
or it wasn't yet understood how an error came to be.

For completeness's sake, it is also mentioned some tools define a checkpoint only as a copy of the storage devices
which may be booted later to revert to such a state.
This definition is not used here.

\subsection{Checkpoints in QEMU}
QEMU itself has a checkpoint mechanism called \emph{savevm}.
These checkpoints are of the desired type, meaning they save the entire state of an emulation,
including CPU registers and memory contents.
However, there are some limitations here, in particular the checkpoints are not stored externally in easily parsable files,
but require a virtual disk for storage, although not all formats are supported here.
This means that the checkpoints would first have to be extracted from these virtual disks for further use,
and then they would have to be parsed to enable use outside of QEMU.
These checkpoints also look different for other architectures, which would likely make it necessary
to implement the parser individually for each architecture.

\section{Related work}
There are multiple scientific works already exploring the possibilities of creating checkpoints from a running QEMU instance,
however they generally attempt to do so by modifying QEMU code or parsing dumps generated by QEMU,
which are not a consistent format.
They still serve as useful information about the internals of QEMU,
especially in regard to the inconsistencies encountered in \autoref{chap:QEMU_API}.

\subsection{\Citetitle{kitcheckpoints}\cite{kitcheckpoints}}
\Citeauthor{kitcheckpoints} has created a tool to extract checkpoints from QEMU running as part of a SimuBoost instance.
Curiously, they have chosen a very different way to achieve their goals,
namely they have chosen to extract the data by modifying the code of QEMU and then using GDB,
a debugger, to extract the data gathered.
In addition, they only used the QEMU Human Monitor (QHM\footnote{\Citeauthor{kitcheckpoints} uses HMP instead of QHM in their thesis, however both terms refer to the same protocol and both are in use.})
in their works, as the QEMU Machine Protocol (QMP) was not nearly as developed as it is now, 11 years later.
The code they developed back then is also nowhere to be found,
making their written thesis the only material to use in regards to his thesis\cite{kitcheckpoints}.

With the time passed since their thesis,
it is also doubtful whether their solution would even work on a modern QEMU instance,
as the ability to merge such old changes into modern software,
which was heavily developed in the meantime,
is usually a very exhaustive process if not downright impossible.
This issue is what this thesis tries to solve,
and while doing so the research by \citeauthor{kitcheckpoints} proved very useful,
especially their explanations of the QEMU memory model\cite{kitcheckpoints}.

\subsection{QPoints\cite{qpoints}}
QPoints is a tool to convert checkpoints from QEMU to Gem5,
as the emulation offered by QEMU is faster when using KVM than the one native to Gem5.
Their goal was to analyze ARM inside Gem5,
and to speed up the process, they created checkpoints from QEMU,
so especially the boot process of an emulation instance could be accelerated.

QPoints is directly targeted at ARM, and while it supports QEMU instances running on different host architectures,
it only supports 64-bit ARM guests and the host must run a UNIX-like operating system like Mac OS or Linux.
It also doesn't support QEMU's QCOW2 disk format, which is effectively the default on modern instances.
It similarly achieves this as the work of \citeauthor{kitcheckpoints},
namely by attaching GDB to the emulation process, fully dumping the data
and then parsing it according to predefined templates.

This generally makes the work by \citeauthor{qpoints} very interesting,
as this thesis would only have to address its core limitation,
namely the lack of support for other architectures,
with the other work they did being very helpful to this thesis.

\subsection{Lapidary\cite{lapidary}}
Lapidary is a tool which was created for the paper \citetitle{lapidary}.
To demonstrate their findings from said paper,
they would have needed some piece of hardware to use as an example.
However, because they did not have time to design and produce such hardware,
they decided to use the Gem5 hardware simulator\cite{gem5} and create checkpoints
they could later use while demonstrating their findings\cite{lapidary_story}.

The tool was developed in Python and extracts data from Gem5 by attaching GDB.
The tool simply takes a full dump of a Gem5 instance,
from which it then extracts all the necessary data.
Sadly however this process is not documented,
as lapidary itself is not mentioned in \citetitle{lapidary}.
but only a short blog article\cite{lapidary_story}
and the codebase\footnote{See \url{https://github.com/efeslab/lapidary}} exist, which however is time-consuming to reverse engineer.

As Gem5 uses its own systems for emulation
while having a rather restrictive application,
the information gathered from this paper was useful for the general concept of this thesis,
however as this thesis uses QEMU, the technical information from \citeauthor{lapidary} remains largely unused.