\chapter{Foundations}
The purpose of this chapter is to lay the groundwork for this thesis
and to establish technical terms used in this writing.
In addition, contradicting definitions get explored
and the process of deciding which ones will be used explained.
The basic technologies involved in this thesis are also adressed in this chapter.

\section{Emulation \& Simulation}
The core of a paper about extracting data from an emulation
to process this data regarding a full system simulation is to fully define these terms,
as they appear very similar.
The importance of a strict definition inside this paper gets even more important
once one starts to actually search for such a definition.
When one starts looking for such a definition,
one will very quickly find one of 2 posts on StackOverflow \cite{SO_link}\cite{SO}.
StackOverflow of course is no reliable source for scientific work,
the reason they are mentioned here is the first one linking to the second one
with the words \enquote{Exactly the opposite answer here:}\cite{SO_link}.
A small discussion, which stays rather civilized for StackOverflow's standards\todonote{Das wieder rausnehmen}, follows,
and people seem to generally agree these terms are heavily subjective and one should clearly define what exactly they're talking about.

This is a small example of showing how there 2 terms are not cleanly defined,
so for this thesis 2 approaches were looked into,
from which a consistent definition inside this thesis is gathered.

\subsection{Emulation \& Simulation according to the Oxford Dictionary}
The Oxford English Dictionary defines the 2 terms as following:
\begin{itemize}
    \item \emph{Emulation}: \enquote{The technique by which a
    computer is enabled, by means of special
    hardware or software, to execute programs
    written for a different type of computer}\cite[p. 196]{emulation_oxford}.
    \item \emph{Simulation}: \enquote{The technique of imitating the behaviour of
    some situation or process (whether economic,
    military, mechanical, etc.) by means of a
    suitably analogous situation or apparatus, esp.
    for the purpose of study or personnel training}\cite[p.503]{simulation_oxford}.
\end{itemize}
From these definititions, one may deduce that emulation is a special case of simulation,
in which an \enquote{apparatus for the purpose of study} is created to \enquote{imitate the behavior of some situation or process},
namely the internal workings of a computer of a different type.\linebreak
The term of \enquote{simulation} instead is easily scaled up or down to suit the current needs.
It may describe a device imitating a large software infrastructure running across many different devices,
perhaps by emulating computers involved in such a system,
however it may also describe the process of imitating the L1-cache of a CPU inside an emulation.\linebreak
This may be concluded by saying that emulation is a special type of simulation,
which itself requires smaller systems of simulation to function,
but can then be used to create bigger systems of simulation.

\subsection{Emulation \& Simulation in scientific literature}
In scientific literature, a search for such a definition does not yield many results.
The closest to a definition in IT comes from \Citeauthor{definition_iot}
while talking about the future chances of the Internet of Things for logistics\cite{definition_iot}.
They explicitly define emulation as a special case of simulation,
and the definitions are generally compatible with the one before.
\begin{itemize}
    \item \emph{Simulation}: A simulator only models and abstracts the system it is simulating.
    This means, in its development decisions must be made as to how precise the simulator works.
    For this, a model must be created and requirements for actual execution developed.
    Later on the results of such a simulation must then be carefully evaluated to make sure
    there are no errors that come from such abstraction
    or are accounted for and get compensated.
    The advantage of such a process is the high speed it carries,
    as simplifications often take away lots of required computation time.
    The disadvantage, as mentioned, however is the potentially high error rate for results,
    and the simulation usually breaking down when trying to analyse small details of a larger system\cite{definition_iot}.
    \item \emph{Emulation}: An emulator is heavily coupled to real components.
    In addition, it's field of work is usually smaller and defined very precisely.
    In return, the results of an emulator are very precise, in most cases exact.
    Error states of the real hardware are also modeled, meaning when feeding it with incorrect data,
    one can observe how the system will behave in abnormal states, which often isn't modeled in a Simulation.
    The big disadvantage of emulation however is its massive cost,
    both in development time and ressources,
    as emulation usually requires a lot more computing power than the device it emulates\cite{definition_iot}.
\end{itemize}

\subsection{Conclusion}\todonote{Think of a better title}
In this thesis therefore Emulation will be viewed as a special kind of simulation,
which perfectly simulates a computing device of a specific architecture
on another computing device, which may or may not be of another architecture.
This especially distinguishes emulation from software like Wine, which is not an emulator\todonote{I really need to stop},
as such software only provides other runtime software a program requires to function, but does not simulate another device in any way\cite{wine}.

In contrast, the abilities of Simulation are much broader,
especially as a simulator may leave out aspects of the real system
its developers don't deem as neccessary for the experiment they want to conduct.
An example of such software in software development is Mockito,
which provides to other parts of a program certain APIs during testing,
which however are not fully implemented yet,
but still return plausible values that allow for testing.

Most people have likely had contact with both these terms in the world of video games,
where they however fulfill completely different jobs.\linebreak
Simulators are usually a type of game,
which allows the player to have an experience he usually does not have in real life.
In a good game, such an experience makes the player feel the imitation is very realistic,
even though it is often simplified heavily.
An example are flight simulators, in which many players feel like they are getting a realistic experience,
even though many features like navigation or communication with Air Traffic Controller are heavily simplified
and can often be turned off.
Which is also an important feature of simulators, the ability to tune the precision
to the current needs to produce a result that is both precise enough for the current problem,
while not wasting resources during uneccessary calculations.\linebreak
Emulators on the other hand get used to allow playing games which were developed for systems which no longer exist.
An example of such an emulator is xemu\cite{xemu},
which we will encounter again later on.
Such emulators perfectly \emph{simulate} the physical hardware of a specific game console,
and allow games compiled for it to run on them.
Xemu for example has to emulate the original Intel Coppermine CPU and Nvidia MCPX X3 GPU
for the games to function, and it must emulate features that may not be needed at all.
For example it always must emulate the SIMD-registers of the CPU, even though a game may not need them.
An additional constraint is performance, the emulator must run at the same speed as the original device,
and any potential performance benefit a modern computer offers is therefore void.
This is because the games are especially developed for such a platofrm,
and the speed of the games is tied to the Frames per Second displayed.
Meaning more frames per second, which usually results in a better experience for the user,
actually cause the game to run faster.

\section{Full System Simulation}
The purpose of this thesis is to produce data which can later on be used inside a \emph{Full System Simulation}.
As however such a simulation is not part of this thesis, this section will only offer a short introduction,
which was mostly lifted from \Citeauthor{kitcheckpoints}\cite{kitcheckpoints}.\todonote{Maybe rephrase this to not make it appear like plagiarism.}

When developing a system, which also requires the development of an operating-system-like
software, the actual hardware may not yet be available for testing
as it often is still in development itself.
Especially when developing embedded devices,
the code develeoped for it often runs on a very low level
and the kernel may take on additional tasks it doesn't on general purpose computers.
To allow for testing of any developed software,
the devices it's supposed to run on must be simulated in its entirety,
with its CPU, RAM, GPU and other external components behaving exactly as they would on the actual device.

Especially such special external components are an issue,
as most emulators provide simplified APIs that are adapted to them being used by a general purpose computer.
For example Hyper-V on Windows does not have the ability to provide a GPU to the guest,
instead Linux has to fallback to \emph{llvmpipe} which provides software rendering.
Other software like VMware and VirtualBox provide heavily simplified GPUs to a guest system,
which are often unable to process 3D rendering by default,
and such a feature must be manually enabled if required.
But even then, there is no option to emulate an actual GPU which may be used on a physical device.