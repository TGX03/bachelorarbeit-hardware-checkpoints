\chapter{Foundations}
The purpose of this chapter is to lay the groundwork for this thesis
and to establish technical terms used in this writing.
In addition, contradicting definitions get explored
and the process of deciding which ones will be used explained.
The basic technologies involved in this thesis are also adressed in this chapter.

\section{Emulation \& Simulation}
The core of a paper about extracting data from an emulation
to process this data regarding a full system simulation is to fully define these terms,
as they appear very similar.
The importance of a strict definition inside this paper gets even more important
once one starts to actually search for such a definition.
When one starts looking for such a definition,
one will very quickly find one of 2 posts on StackOverflow \cite{SO_link}\cite{SO}.
StackOverflow of course is no reliable source for scientific work,
the reason they are mentioned here is the first one linking to the second one
with the words \enquote{Exactly the opposite answer here:}\cite{SO_link}.
A small discussion, which stays rather civilized for StackOverflow's standards, follows,
and people seem to generally agree these terms are heavily subjective and one should clearly define what exactly they're talking about.

This is a small example of showing how there 2 terms are not cleanly defined,
so for this thesis 2 approaches were looked into,
from which a consistent definition inside this thesis is gathered.

The Oxford English Dictionary defines the 2 terms as following:
\begin{itemize}
    \item \emph{Emulation:} The technique by which a
    computer is enabled, by means of special
    hardware or software, to execute programs
    written for a different type of computer\cite[p. 196]{emulation_oxford}.
    \item \emph{Simulation}: The technique of imitating the behaviour of
    some situation or process (whether economic,
    military, mechanical, etc.) by means of a
    suitably analogous situation or apparatus, esp.
    for the purpose of study or personnel training\cite[p.503]{simulation_oxford}.
\end{itemize}
From these definititions, one may deduce that emulation is a special case of simulation,
in which an \enquote{apparatus for the purpose of study} is created to \enquote{imitate the behavior of some situation or process},
namely the internal workings of a computer of a different type.\linebreak
The term of \enquote{simulation} instead is easily scaled up or down to suit the current needs.
It may describe a device imitating a large software infrastructure running across many different devices,
perhaps by emulating computers involved in such a system,
however it may also describe the process of imitating the L1-cache of a CPU inside an emulation.\linebreak
This may be concluded by saying that emulation is a special type of simulation,
which itself requires smaller systems of simulation to function,
but can then be used to create bigger systems of simulation.