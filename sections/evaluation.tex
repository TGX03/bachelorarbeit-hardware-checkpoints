%% LaTeX2e class for student theses
%% sections/evaluation.tex
%% 
%% Karlsruhe Institute of Technology
%% Institute of Information Security and Dependability
%% Software Design and Quality (SDQ)
%%
%% Dr.-Ing. Erik Burger
%% burger@kit.edu
%%
%% Version 1.5, 2024-02-12

\chapter{Evaluation}
\label{chap:Evaluation}
Evaluating the work of this thesis was no trivial task.
The most important part of an evaluation is obviously the correctness of the gathered results,
if the results of any program are not correct, there is no point in analyzing its performance or flexibility.

A very interesting approach here would of course have been to use the tool developed by \citeauthor{kitcheckpoints}
and to compare the results of each solution.
This especially would have offered the possibility to compare two very different approaches,
as \citeauthor{kitcheckpoints} directly modified QEMU's code to achieve a goal similar to the one of this thesis,
while this thesis aims to do so without manipulating QEMU in any way to achieve better compatibility.
Such a comparison would not only have been interesting regarding the correctness of either solution,
but especially in regards to performance and resource usage,
as one might suspect \citeauthor{kitcheckpoints} solution to be better in these regards,
as my solution uses Java and JSON, both a lot slower tools when compared to C and SQL which were used by \citeauthor{kitcheckpoints}.
Unfortunately, I could not find the original code \citeauthor{kitcheckpoints} wrote for his paper,
as they did not link any kind of such a repository in his paper.
Additionally, one may question whether it would even be possible to get his code to work 11 years after publication,
as QEMU has of course changed massively in the meantime, which may have broken the modifications they made to QEMU.
Because of this, it was not possible to perform such a direct comparison.
It was also refrained from using his original numbers,
as I do not have similar hardware available,
making any kind of performance comparison obsolete,
especially when considering the technological advancements of the last decade\cite{kitcheckpoints}.

Another way to verify any extracted data would have been to re-inject them into a running QEMU instance and see if the result is valid.
However, developing such a mechanism is not the direct goal of this thesis,
and when looking at the commands QMP and QHM offer, it is also evident such a task is not the purpose of these APIs.
This means, such a reinjection could likely be its own paper,
and would thereby throw this chapter completely out of proportion
and make it a thesis inside a thesis.
\eqnuote{Re-Injection} is therefore reserved as a topic for future work in \todonote{Link future work}
and will not be discussed further for the purposes of evaluating this thesis.

On the point of specifications, the system used for any kind of testing consists of the following parts:
\begin{itemize}
    \item CPU: AMD Ryzen 9 7900X with overclocked all-core boost of 5.5GHz
    \item RAM: 48GB DDR5-7000 CL36
    \item GPU\footnote{None of the tests require a GPU, only included for completeness}: Nvidia Geforce RTX 3090 overclocked to 2.1GHz
    \item Hard Drives:
    \begin{itemize}
        \item Samsung SSD 990 Pro 1TB
        \item Samsung SSD 970 Evo Plus 1TB
        \item 2* Western Digital Blue 1TB in Raid 0 cached by the 990Pro
    \end{itemize}
    \item Operating System: Windows 11 Pro
\end{itemize}
This is not a clean Windows install, but instead many tools a developer may use,
like IntelliJ, VS Code, Mircosoft Office and so on are running on this machine.

\section{Correctness \& Completeness}
As such, the first thing to discuss is whether the results produced actually hold up.
To achieve this, the results are discussed individually for each region in \autoref{chap:implementation}.
The main issue of this section is the lack of any alternative software to gather the same data for comparison.
While there are tools like QEMU's internal checkpointing mechanism,
to compare the data gathered one would first need to parse QEMU's results and the results of this paper,
which brings multiple possibilities for errors not directly related to this thesis itself,
especially as QEMU's checkpoints are stored in a serialized format,
while this thesis aims to use a general purpose format, namely JSON.
To work around this, for each kind of data point,
solutions were found and any compromises made are mentioned and justified.

\subsection{CPU}\label{sec:eval_cpu}
When developing the extraction for the CPU,
the main problem was to work around all the oddities described in \autoref{sec:info_registers}.
The data itself was however externally accessible in a human-readable form,
which means, under the assumption that QEMU provides correct data through this API,
the correctness of the data only depends on whether the parser is implemented correctly.
To verify the extracted data, one must now somehow compare the output of the \emph{info registers} command
with the data present in the internal model of the developed solution.

To automate such a process, one would have to implement another parser which parses \emph{info registers}
and then compare the output of both parsers and check whether they are correct.
This approach is rather ridiculous, as it assumes when writing the same program 2 times,
one of them must be correct.

Therefore, to not write the same code with the same errors twice,
this was done manually for ARM, x86 and x64 during development,
and for these architectures the parsing now works correctly.
There are of course different architectures which contain different data
and may format the data differently, however the basic structure is always similar,
meaning the developed solution always extracts its data correctly.

Still, when operating on other architectures, for example RISC-V or Power-PC,
special care should be attributed to verifying
the results produced by this thesis' developments
correctly handle such architectures, especially in the beginning to avoid exhaustive reworks later on.

\subsubsection*{Completeness}
While the data extracted is correct, the bigger question here is whether all data is extracted.
On ARM, for example, there may exist the line \enquote{FPU disabled}.
Such plaintext is not parsed at all,
as there is no easy resource containing all such special data points one may encounter,
and therefore they are not implemented.
Additionally, the tool only parses a single flag array reported by \emph{info registers} per CPU,
meaning any flags not contained in the first registers do not get included in a checkpoint.
On x86, this means the registers associated with the pointer registers,
containing information about the access restrictions of the referenced memory region,
are currently not included at all in a checkpoint.

If one intends to use the checkpoints produced by this thesis for their own purposes,
it is therefore imperative they check whether the data included in a checkpoint is sufficient for their work.
Especially analyzing the flags of any CPU requires special care
and perhaps additional modifications of the code produced by this thesis.

\subsection{System memory}
The biggest part in the implementation phase of this thesis
was the extraction of data from the guest's memory.
This also warrants special attention in the evaluation of this data,
as all of the approaches attempted faced some issues during implementation.

In addition, this the only one of the three cases
in which the data provided by QEMU cannot be used as-is.
Instead, a third-party format is used between QEMU and the extraction,
which is implemented using another library
which is not maintained by a big organization
and does not seem to be used in a widespread manner.

At the same time, the amount of data to check for consistency is rather large,
multiple gigabytes in most cases.
Like in \autoref{sec:eval_cpu}, this makes the easiest way of verifying the correctness of the data
to directly compare it with the output of QEMU.
For this, the QHM commands discussed in \autoref{sec:memsave} seem like good candidates.
Since the tool developed in this thesis produces multiple dump files
in relation to the memory regions included in the original ELF file,
those commands can be used to extract the exact same regions
and then compare the contents of these files.
The code of this test is shown in \autoref{fig:test_memory}

\begin{figure}[h]
    \begin{ffcode}
        QMPInterface inter = new QMPInterface("localhost", 4444);
        inter.executeCommand(Stop.INSTANCE);
        ELFDump elf = new ELFDump(inter);
        inter.executeCommand(elf);
        elf.awaitCompletion();
        for (MemorySegment segment : elf.getSegments()) {
            long startAddress = segment.getStartPhysicalAddress();
            long size = segment.getSize();
            Path target = Paths.get("D:\\Lars\\out\\" + startAddress + ".dmp");
            PMemsave save = new PMemsave(startAddress, size, target);
            inter.executeCommand(save);
            InputStream elfInput = segment.getInputStream();
            try (InputStream dumpInput = Files.newInputStream(target)) {
                int elfRead;
                do {
                    elfRead = elfInput.read();
                    int dumpRead = dumpInput.read();
                    if (elfRead != dumpRead) System.out.print("Found mismatch");
                } while (elfRead != -1);
            }
        inter.executeCommand(Continue.INSTANCE);
  \end{ffcode}
  \caption{The code used to check for correct extraction of memory contents.}
  \label{fig:test_memory}
\end{figure}

As can be seen, this code writes \enquote{Found mismatch} to the standard output
if any discrepancies between the two files were found.
This test did in fact unearth some issues with the code,
mainly related to the implementation of an Input Stream to read from a 2D array
as discussed in \autoref{sec:Java_2GB},
but in the end, all of the data was consistent.

\subsubsection*{Inconsistency during the development cycle}
At one point during early testing,
it was observed most of the memory dumps started with the same group of bytes.
At that time however, no bigger focus was placed on this curiosity,
as there were still other issues waiting to be solved.
Later on, this behavior could not be replicated anymore,
even though no explicit effort was taken before to address it.
It is possible this \enquote{bug} was solved by accident
or while fixing another part of the software,
however such oddities should be mentioned here to make others aware of them and look out for them.

\subsubsection*{Deduplication}
The tool offers the ability to deduplicate checkpoint data,
however it is nowhere near as sophisticated as the solution offered by \citeauthor{kitcheckpoints}.
Currently the tool has no ability to load comparison data from disk,
thoug such a process would be trivial to implement,
as only the created JSON needs to be parsed.
Still, this results in the tool only deduplicating checkpoint data
against checkpoints created previously by the same process.
This can be observed by the amount of memory files created
when creating follow-up checkpoints.
The more time has passed between 2 checkpoints,
the more memory dump files get stored in the checkpoint directory.
In addition, when the emulator is paused and does not run between 2 checkpoints, no new memory files get created.

\subsection{block devices}
As the implementation of block devices was again rather trivial,
there is no explicit task to test here.
The tool simply copies the files from the original location provided by QEMU
to the destination desired by the user.
And, if we assume the copy process provided by Java works,
it can be assumed this step is fully functional.
There may be errors in regards to permissions of the file to copy,
however that is not something this thesis has to look at.

\subsubsection*{Deduplication}
The approach used for deduplicating block devices is essentially the same as the one for memory dumps.
The code is nearly identical, as the comparison of hashes for both types is the same.
This also allows for the assumption, if the deduplication works for memory regions,
it also works foblock devices.
The main difference is the rate of follow-up checkpoints is much lower,
block devices generally experience fewer changes than memory regions.
Depending on the use case, this can however differ as block devices are copied as a whole and not in parts like main memory.