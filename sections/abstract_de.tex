%% LaTeX2e class for student theses
%% sections/abstract_de.tex
%% 
%% Karlsruhe Institute of Technology
%% Institute for Program Structures and Data Organization
%% Chair for Software Design and Quality (SDQ)
%%
%% Dr.-Ing. Erik Burger
%% burger@kit.edu
%%
%% Version 1.5, 2024-02-12

\Abstract
Während der Softwareentwicklung wird häufig Emulation als Technik eingesetzt,
um Software für Hardware zu entwickeln, welche noch nicht existiert oder parallel zur Software entwickelt werden soll.
Das derzeit am häufigsten verfügbare Tool für eine solche Emulation ist QEMU, welches eine Vielzahl von Hardwarearchitekturen unterstützt.
Während eine Emulation in QEMU ausgeführt wird,
existieren viele Daten innerhalb von QEMU, die für eine weitere Untersuchung außerhalb einer laufenden QEMU-Emulation nützlich sein könnten.
Diese Arbeit extrahiert daher die Daten einer solchen Emulation und macht sie für die weitere Verwendung zugänglich.
Um dies zu erreichen, werden das QEMU-Maschinenprotokoll und der QEMU-Monitor verwendet, um diese Daten aus einer laufenden QEMU-Instanz zu extrahieren,
ohne den QEMU-Code zu ändern, da dies oft ein langer Prozess ist,
nicht sehr flexibel ist und auch umfangreiche Nacharbeiten erfordern kann, wenn sich die QEMU-Codebasis ändert.
Darüber hinaus gibt dieses Dokument Einblicke in die internen Vorgänge der Emulation und diskutiert Probleme damit.
Derzeit extrahiert diese Arbeit Daten über CPU, Speicher und angeschlossene Speichergeräte aus jeder QEMU-Instanz, die QMP unterstützt.
Die Ergebnisse dieser Arbeit sind verfügbar unter \url{https://github.com/TGX03/bachelorarbeit-hardware-checkpoints}