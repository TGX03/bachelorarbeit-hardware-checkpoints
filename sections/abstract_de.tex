%% LaTeX2e class for student theses
%% sections/abstract_de.tex
%% 
%% Karlsruhe Institute of Technology
%% Institute for Program Structures and Data Organization
%% Chair for Software Design and Quality (SDQ)
%%
%% Dr.-Ing. Erik Burger
%% burger@kit.edu
%%
%% Version 1.5, 2024-02-12

\Abstract
Während der Emulation eines Computers oder vergleichbarer Geräte, wie Smartphones oder Videospielkonsolen,
liegen viele Daten innerhalb der Emulationsinstanz vor, die extern für weitere Forschungen verwendet werden können.
Diese Arbeit versucht daher, die Daten einer solchen Emulation zu extrahieren und für die weitere Verwendung zugänglich zu machen.
Um dies zu erreichen, werden das QEMU Machine Protocol (QMP) und der QEMU-Monitor (QHM) verwendet, um diese Daten aus einer laufenden QEMU-Instanz zu extrahieren,
ohne den QEMU-Code zu modifizieren, um so die Funktionalität über mehrere QEMU-Versionen hinweg zu erhalten.
Darüber hinaus gibt dieses Dokument Einblicke in die internen Vorgänge der Emulation und diskutiert damit verbundene Probleme.
In der Entwicklung sind nämlich diverse solcher Probleme aufgetreten,
aufgrund welcher diese Arbeit derzeit nur die Möglichkeit, Daten über CPUs, RAM und angeschlossene Speichergeräte aus einer QEMU-Instanz zu extrahieren, bietet.
Sie ist derzeit außerdem auf maximal 2 GiB Gastspeicher beschränkt, und hat dabei einen erheblichen Ressourcenverbrauch.