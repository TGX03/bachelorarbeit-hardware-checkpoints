%% LaTeX2e class for student theses
%% sections/abstract_de.tex
%% 
%% Karlsruhe Institute of Technology
%% Institute for Program Structures and Data Organization
%% Chair for Software Design and Quality (SDQ)
%%
%% Dr.-Ing. Erik Burger
%% burger@kit.edu
%%
%% Version 1.5, 2024-02-12

\Abstract
Die Emulation von Hardware, die nicht von nativer Emulation wie VT-x oder AMD-v profitiert, führt zu enormen Leistungseinbußen.
Um dieser Verlangsamung entgegenzuwirken, werden Prüfpunkte aus bereits laufenden Emulationen erzeugt,
die in einem verallgemeinerten Format gespeichert werden um eine weitere Verarbeitung zu ermöglichen.
Um dies zu erreichen, werden das QEMU Machine Protocol und der QEMU Human Monitor verwendet, um Daten aus einer laufenden QEMU-Instanz zu extrahieren,
wobei versucht wird, Änderungen am QEMU-Code zu vermeiden, um die Funktionalität über mehrere QEMU-Versionen hinweg beizubehalten.
Darüber hinaus gibt dieses Dokument Einblicke in die internen Vorgänge der Emulation und diskutiert damit verbundene Probleme.
Ursprünglich war auch geplant, dass in dieser Arbeit eine Re-Injezierung dieser Daten in eine laufende QEMU-Instanz entwickelt wird,
jedoch stellte sich diese Aufgabe als zu groß für ein Projekt diesen Rahmens heraus.
Stattdessen werden deswegen nur grundlegende Ansätze für einen solchen Vorgang erläutert.