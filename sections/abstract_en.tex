%% LaTeX2e class for student theses
%% sections/abstract_en.tex
%% 
%% Karlsruhe Institute of Technology
%% Institute of Information Security and Dependability
%% Software Design and Quality (SDQ)
%%
%% Dr.-Ing. Erik Burger
%% burger@kit.edu
%%
%% Version 1.5, 2024-02-12

\Abstract
Emulation of hardware that does not profit from native emulation like VT-x or AMD-v carries an enormous performance penalty.
To combat this slowdown, checkpoints get created from already existing emulations,
which get stored in a generalised format to allow for further processing.
To achieve this, the QEMU  machine protocol and the QEMU monitor get used to extract data from a running QEMU instance,
while trying to avoid modifying QEMU-code to keep functionality across multiple QEMU versions.
In addition, this paper gives insights into the internals of emulation and discusses issues with them.
Initially, this paper also aimed to include a way of reinjecting such gathered checkpoints into a running instance of QEMU,
however this turned out to be too difficult for a project of this size,
and instead only the basic outlines of such a process are discussed.