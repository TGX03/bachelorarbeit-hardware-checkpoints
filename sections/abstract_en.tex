%% LaTeX2e class for student theses
%% sections/abstract_en.tex
%% 
%% Karlsruhe Institute of Technology
%% Institute of Information Security and Dependability
%% Software Design and Quality (SDQ)
%%
%% Dr.-Ing. Erik Burger
%% burger@kit.edu
%%
%% Version 1.5, 2024-02-12

\Abstract
During software development, emulation is a technique often used when developing for hardware solutions which do not yet exist
or are being developed in parallel with the software.
The mian tool for such emulation currently available is QEMU, offering support for a wide variety of hardware architectures.
While running an emulation inside QEMU,
a lot of data exists inside QEMU which may be useful for further inspection outside of a running QEMU emulation.
This thesis therefore extracts the data of such an emulation and makes it accessible for further use.
To achieve this, the QEMU machine protocol and the QEMU monitor get used to extract this data from a running QEMU instance,
without modifying the QEMU code, which is often a long process,
not very flexible and may also require extensive rework when the QEMU codebase changes.
In addition, this paper gives insights into the internals of emulation and discusses issues with them.
Currently, this thesis extracts data about CPU, memory and attached storage devices from any QEMU instance supporting QMP.