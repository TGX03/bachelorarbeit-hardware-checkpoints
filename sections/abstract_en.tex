%% LaTeX2e class for student theses
%% sections/abstract_en.tex
%% 
%% Karlsruhe Institute of Technology
%% Institute of Information Security and Dependability
%% Software Design and Quality (SDQ)
%%
%% Dr.-Ing. Erik Burger
%% burger@kit.edu
%%
%% Version 1.5, 2024-02-12

\Abstract
During emulation of a computing device,
a lot of data exists inside the emulation instance which may be used externally to perform further research.
This thesis therefore tries to extract the data of such an emulation and make it accessible for further use.
To achieve this, the QEMU machine protocol and the QEMU monitor get used to extract this data from a running QEMU instance,
whithout modifying QEMU-code to keep functionality across multiple QEMU versions.
In addition, this paper gives insights into the internals of emulation and discusses issues with them.
There were many issues encountered a long the way,
so currently this thesis only offers to extract data about CPUs, RAM and attached storage devices from a QEMU instance.
It is also currently limited to a maximum of 2GiB of guest memory and has a considerable resource usage.